%!TEX program = xelatex
%!TEX TS-program = xelatex
%!TEX encoding = UTF-8 Unicode

\documentclass[12pt, a4paper]{article} % A4 纸,字体大小为 12pt 的 article 类文档
\usepackage{CJKutf8} % 中文支持
\usepackage{graphicx} % 插入图片
\usepackage{subfigure} % 插入多图时用子图显示的宏包
\usepackage{listings} % 支持代码显示
\usepackage[colorlinks,linkcolor=blue]{hyperref} % 超链接
\usepackage{ulem} % 删除线
\usepackage{xcolor} % 定制颜色
\usepackage{caption2} % 浮动图形和表格标题样式
\usepackage{amssymb} % 数学符号
\usepackage{indentfirst} % 中文段落首行缩进
\usepackage{tikz} % 画图
\usepackage{pgfplots} % 画图
\usepackage{amsmath} % 处理数学公式
\usepackage{mathtools} % 处理数学公式
\setlength{\parskip}{0.5em} % 段落间距
\renewcommand{\figurename}{图} % 将图表的标题设置为中文“图”
\usetikzlibrary{tikzmark,calc,decorations.pathreplacing} % tikzmark 用于标记位置,calc 用于计算,decorations.pathreplacing 用于画大括号


\title{第二十一 · 机制设计 · 青出于蓝如何胜于蓝}
\author{hoochanlon}
\date{\today}

\begin{document}
	\begin{CJK*}{UTF8}{gbsn}
		\maketitle
        \clearpage
        \section{机制设计}
        信息甄别,指鹿为马,通过故意露出破绽,来筛选其指定的人选。指鹿为马是信息甄别小技巧,而机制设计更能体现思想性。
        \subsection{机制设计理论}
        对于任意给定的一个目标,在自由选择、自愿交换的分散化决策条件下,能否并且怎样设计一个合理机制,使得经济活动参与者的个人利益和设计者所要实现的目标一致。
        我们可以把机制设计理论视为 博弈论的反向操作。\par
        博弈论是给定某个游戏规则,分析每个人如何做出自己的策略和行为选择,得出什么博弈结果是纳什均衡。那么机制设计呢,它是一种顶层设计的逻辑,
        给定某个我们想要的结果,然后找到一个能够实现这个结果游戏规则。在游戏规则下面,每个人追求自己利益最大化的策略和行为选择。导致那个我们想要的结果,
        恰好是所有人博弈的纳什均衡解。\par

        任何一个经济机制的执行,都需要信息传递。而信息传递是需要成本。因此对于机制设计者来说,信息空间维数是越小越好。机制设计从信息的观点出发,
        把经济机制看作是一个信息交换和调整的过程。从各种经济机制的信息成本角度,来比较机制的优和劣。赫尔韦茨在70年代的研究成果证明,在纯交换的新古典经济环境中,
        竞争的市场机制,用最少的信息,就达到了有效的资源配置。换句话说,用市场配置资源的信息成本是最低的,配置结果最有效。
        不同的制度竞争,本质上是效率的竞争,不同组织间的竞争,也是如此。学习成绩上的竞争,某种程度上也是学习效率的一种竞争。\par

        在社会经济活动中,通常机制设计者的目标和机制参与者的利益之间往往不是完全一致的,要达到机制设计者的某种目标,就必须对参与者给予激励,机制参与者
        只有能获得大于其付出代价的利益时,才会遵循该机制的约束和要求,才会把事情做好。否则,他们就会选择不遵循该机制的约束,或者不把事情做好。因此对于
        机制设计者来说,要想实现目标,必须建立合理有效的激励机制。 \par

        \[
            \text{机制设计}
            \begin{cases}
                 \text{信息效率问题}\left\{
                    \begin{aligned}
                        &\text{关于经济机制实现,既定目标所要求的}  \\
                        &\text{信息量多少的问题。即机制运行的成本问题} \\
                    \end{aligned}
                 \right. \\\\
                 \text{激励相容问题}\left\{
                    \begin{aligned}
                        &\text{无论机制设计者想要实现的目标是什么,}  \\
                        &\text{其所设计的机制或规则,必须建立每个人} \\
                        &\text{都追求个人利益最大化的前提下,借助于}  \\
                        &\text{每个人的理性选择来实现的。}
                    \end{aligned}
                 \right.
            \end{cases}
        \]

        所谓激励相容,简单说就是我希望你做的事情,正是你最愿意做的事情。从日常生活中来看,可以简单理解为说谎和偷懒问题。信息不对称导致参与人事情隐藏信息,
        事后隐藏信息或事后隐藏行动。那么说谎可以说是隐藏信息,偷懒就是隐藏行动;激励相容就是要通过机制设计,让参与者觉得说真话,努力工作,不说谎不偷懒,
        整体行为对自己是最有利的。 \par

        当然做到以上并不容易,而且往往需要付出相应的成本代价。在信号博弈传递过程中,参与者需要花成本来证明自己的类型;在信息甄别中参与者花需要成本,
        来区分其他人的类型。那么在拍卖理论中,第二高价封标拍卖就是用最高价和第二高价的差价,来诱惑买家报出自己的真实私人价值。
        因此我们可以把拍卖理论理解为机制设计理论的某种具体应用。 \par

        \subsection{评价经济机制优劣的基本标准}

        机制设计理论的核心诉求是如何设计出一种好的经济制度,能同时实现以上三个要求。机制设计的重点,放在如何在信息分散和信息不对称的条件下,
        设计出更具体的激励相容机制,来实现资源的有效配置。

        \begin{itemize}
            \item[] \textbf{资源的有效配置} 通常采用帕累托最优标准
            \item[] \textbf{信息的有效利用} 要求机制运行花费尽可能低的信息成本。
            \item[] \textbf{激励相容} 要求个人理性和集体理性一致。
        \end{itemize}

        \subsection{马斯金和迈尔森主要研究成果}
        显示原理厉害之处,它通过给出的一般性机制与报告真实信息的直接机制进行等价。使人们可以将注意力集中在报告真实信息的直接机制上来,这就使得很多问题可以进行有效数理方法处理。
        显示偏好原理不能解决多均衡问题,激励相容保证每个人说真话是一个均衡,但不能保证这个均衡是唯一的均衡。如何设计机制使所有均衡结果对给定目标而言,都是最优的,合理的和重要的,这就是执行问题。\par

        马斯金把博弈论引入机制设计和执行理论,认为机制设计并不需要一个中央的计划执行者,顶层设计这件事情并不需要高高在上的主导者。在非合作博弈中,
        每个参与者在考虑自己利益的时候会按照机制设计者的意图行动,从而实现机制设计所要达到的目标。


        \begin{itemize}
            \item 显示原理:无论哪种资源配置,如果能够被某个机制所实现,就一定存在一个直接机制,并且在这一直接机制中,每个理性参与人都会真实报告自己的信息。
            \item 执行理论:研究强制执行法律制度的学科。
        \end{itemize}

        \clearpage
        \section{委托代理理论}
        所有权与控制权相分离:随着公司规模日趋扩大,公司股权日趋分散,分散的股权使得所有权与控制权日趋分离。那么在现代公司中,公司的控制权实际上已经落入到管理者手中,
        那么经营管理权和所有权的分离,促成一场企业权利从股东手中转移到了经理阶层手中的经理革命。企业的股东发现,他们的利益和管理者不一致,两者存在利益冲突。
        对于所有者来说其动机在于获得最大资产收益;对于管理者来说,其行为动机多元,除了获取个人经济收入动机以外,还包括提高自己的社会经济地位,扩大资源调度能力,
        以及实现自我价值的一些动机。\par

        管理者行为动机的多元性决定了其行为目标的多元化,不仅有经济收入的目标,还有名誉、地位、权势等目标。就必然存在股东们如何激励管理者,为自己努力工作的一个现实问题。
        从博弈论的视角来看,由于是所有者和管理者之间,存在严重信息不对称。所有者无法有效观测管理者的努力程度,再加上企业的经营绩效,不是完全对应关系。加上市场受外部影响,
        那么管理者可以把糟糕的业绩,归咎为恶劣市场环境。因此,所有者就需要考虑如何通过设计一个薪酬合约,对管理者进行有效激励,以诱使其按照所有者的目标来行事,这就是所谓的委托代理的问题。\par

        委托代理的关键就是合约设计,所以委托代理理论也被称为叫合同理论,其本质就是一个建立机制的设计问题。委托代理不仅仅局限于企业经营,还广泛存在于社会生活的方方面面。
        \textbf{只要信息不对称,就必然存在委托代理关系}。但在委托代理中,就是代理人(信息优势的一方)的私人信息(包括知识和行动),它会影响到委托人
        (就是信息劣势的这一方)的利益,那么委托人就必须为自己的无知去买单,这被称为叫代理成本。\textbf{委托代理理论所要解决的核心问题就是,如何与信息优势的一方打交道}\par

        \subsection{委托代理的博弈过程}
        \begin{enumerate}
            \item 委托人设计一个合同。
            \item 代理人根据自己的真实情况选择是否接受该合同。
            \item 代理人依据合同提供一定的努力水平。
            \item 随机因素和代理人的努力水平共同决定了某种结果。
            \item 委托人根据结果和合同中的具体约定完成支付。
        \end{enumerate}

        \subsection{委托代理博弈模型化}
        注:效用函数是描述个体或决策者对不同选择或结果的偏好程度的数学函数。它通常用于经济学和决策理论中,在分析人们的决策行为和评估各种决策方案的效果时发挥重要作用。
            \begin{itemize}
                \item 令A为代理所有可选择的行动组合。
                \item a表示代理人的特定行动,且 $ a \in A$
                \item $\Theta$ 代表不受代理人和委托人控制的外生变量。
                \item $G(\theta)$表示 $\theta$在$\Theta$的分布函数。
                \item $g(\theta)$表示 $\theta$在$\Theta$的密度函数。
            \end{itemize}
            a和$\theta$共同决定一个可以观测的结果x,x的函数就是$x(a,\theta)$;$\pi(a,\theta)$表示货币收入(归委托人所有)。
            \begin{itemize}
                \item[] 假设 $\pi(a,\theta)$ 是a的严格递增凹函数 $(\pi^{\prime}>0, \pi^{\prime\prime}<0)$。
                \item[] c(a)代表代理人的成本 $(c^{\prime}(a)>0,c^{\prime\prime}(a)>0)$
                \item[] s(x)代表委托人对代理人支付函数。
                \item[] $V(\pi-s(x))$表示委托人的期望效用函数。
                \item[] $U(s(x))-c(a)$ 表示代理人的期望效用函数。
                \item[] $V^{\prime}>0, V^{\prime\prime} \leq 0, U^{\prime}>0, U^{\prime\prime} \leq 0$
                \item[] $\pi^{\prime} > 0$和$c^{\prime} > 0$ 对委托人来说代理人越努力收益越大,而对代理人成本变高了。
            \end{itemize}

            委托人需要设计一个最优激励合同s(x),来充分激励代理人努力工作。这样既实现了代理人的利益最大化,又实现了委托人利益最大化。
            那么委托代理人的建模方法,有三种:状态空间模型化方法、分布函数参数化方法、一般分步方法。

            \subsubsection{*状态空间模型化方法}
            假设: $V(\pi-s(x))$表示委托人的期望效用函数;$U(s(x))-c(a)$ 表示代理人的期望效用函数。\par
            \begin{itemize}
                \item[] P: $\int v((\pi-s(x)) - u(s(x)))g(\theta)d(\theta)$
                \item[] IR: $\int u(s(x(a,\theta)))g(\theta)d(\theta) - c(a) \geq \bar{u}$
                \item[] IC:
                \begin{align*}
                    \int u(s(x(a,\theta)))g(\theta)d\theta - c(a) &\geq \int s(x(a',\theta))g(\theta)d\theta-c(a')\\
                     &\quad \forall a' \in A
                    \end{align*}
            \end{itemize}
            在自然条件下面,代理人履行合同责任后所获得的收益,不能低于他的期望收益。这就是IR,即代理人的最低收益保证。
            IC是代理人的激励相容性约束,即代理人的最高收益保证,代理人在获得预期效用最大化的同时,然后保证委托人他预期收益也能够最大化。\par
            \begin{align*}
                \max \int v((\pi-s(x)) - u(s(x)))g(\theta)d(\theta)\\
                s.t. \int u(s(x(a,\theta)))g(\theta)d(\theta) - c(a) \geq \bar{u}\\
                \int u(s(x(a,\theta)))g(\theta)d\theta - c(a) &\geq \int s(x(a',\theta))g(\theta)d\theta-c(a')\\
                &\quad \forall a' \in A
                \end{align*}
                委托人的问题是满足代理人约束与激励相容性约束的前提下,选择一个什么样的a和s(x)。委托人的目标是最大化自己的效用函数,即最大化P。

            \clearpage
            \section{若干研究成果和应用}
            \subsection{不同信息条件下的博弈均衡}
            在完全信息博弈下面,代理人的行为是可以被观察到的,委托人可以根据观测到的代理人行为,直接对其实行奖惩。那么这时,双方的博弈结果能够确保
            是帕累托最优风险分担和帕累托最优努力水平。\par

            在信息不对称的情况下面,由于代理人的行为是不能被完全观测到的,而且产出又受到外部环境的影响。那么委托人在对代理人进行奖惩,那么就不能保证同时达到
            最优风险分担和最优努力水平。比如:企业对销售的底薪激励。如果底薪太高,那相应的提成比例就低,那么销售员就得不到充分激励;如果底薪太低,销售员又不愿意承担过大的风险。
            高底薪高提成,不符合委托人利益最大化,企业不愿意,,因此就需要在激励和风险两方面做权衡。

            \subsubsection{声誉模型}
            在竞争性的经理人市场下面,经理人的市场价值往往取决于他过去的经营业绩。从长期来看,经理人对自己的行为是负完全责任的。
            如果我们将委托代理关系扩充到长期的多阶段重复博弈,那么代理人就必须关注未来收入,那么这样的话,即使没有显性的激励机制,代理人也会努力工作。
            改进自己的声誉,由此提高自己的收入水平。

            \subsubsection{棘轮效应}
            代理人越是努力,好业绩可能性越大,自己给自己的“标准”就定的越高。当代理人意识到努力带来的结果是“标准”的提高,代理人的努力积极性就越低。(鞭打快牛)

            \subsubsection{锦标效应}
            如果都从事相同工作,一个代理人的工作结果,能够提供另外一个代理人的工作信息,那么代理人工资不仅要依赖自己的产出,还要考虑其他代理人的产出。(相对业绩评估)
            目的排除外部影响及其带来的不确定性,让代理人的努力程度达到更直观的体现。在锦标制度中,代理人的所得跟绝对表现没有关系,而是跟相对一群人整体有关。
            在基本的委托代理模型中它不是最优的,但它有其自身的优势,比如操作方便,而且有利于解决委托人的道德风险。

            \subsubsection{强制退休}
            在长期的雇佣关系中,工龄工资可以遏制偷懒的行为。雇员在早期阶段的工资,低于边际生产率,两者的差距等于是交了一种保证金。当偷懒被发现时雇员被开除,
            就损失了保证金,因此就提高了偷懒成本,并提高了努力工作的积极性,而且该模型也同时解释了强制退休的原因:到一定阶段及一定年龄以后,雇员的工资会大于
            边际生产率,当然就不会有人主动退休了。

            \subsubsection{多项任务模型}
            当代理人从事多项任务工作的时候,从简单的委托代理模型得出的结论是不适用的。固定工资合同可能是优于根据可观察的业绩变量奖惩代理人的激励合同。
            当代理人从事多项工作的时候,对任何给定工作的激励,不仅仅取决于该工作本身的可观察性,还取决于其他工作的可观察性。特别是如果委托人期待代理人
            在某项工作上面花费一定的精力,而这项工作又不可观测的时候,那么激励工资其实并不是一个好的激励机制。比如大学教师工资,教学质量比科研成果难以被观测,
            对科研激励反而会损害老师们的教学质量。由此固定工资反而是更好的。





    \end{CJK*}
\end{document}
