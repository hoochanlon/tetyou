%!TEX program = xelatex
%!TEX TS-program = xelatex
%!TEX encoding = UTF-8 Unicode

\documentclass[12pt, a4paper]{article} % A4 纸,字体大小为 12pt 的 article 类文档
\usepackage{CJKutf8} % 中文支持
\usepackage{graphicx} % 插入图片
\usepackage{subfigure} % 插入多图时用子图显示的宏包
\usepackage{listings} % 支持代码显示
\usepackage[colorlinks,linkcolor=blue]{hyperref} % 超链接
\usepackage{ulem} % 删除线
\usepackage{xcolor} % 定制颜色
\usepackage{caption2} % 浮动图形和表格标题样式
\usepackage{amssymb} % 数学符号
\usepackage{indentfirst} % 中文段落首行缩进
\usepackage{tikz} % 画图
\usepackage{pgfplots} % 画图
\usepackage{amsmath} % 处理数学公式
\usepackage{mathtools} % 处理数学公式
\setlength{\parskip}{0.5em} % 段落间距
\renewcommand{\figurename}{图} % 将图表的标题设置为中文“图”
\usetikzlibrary{tikzmark,calc,decorations.pathreplacing} % tikzmark 用于标记位置,calc 用于计算,decorations.pathreplacing 用于画大括号


\title{第二十·拍卖理论·等价收入关联最终收益}
\author{hoochanlon}
\date{\today}

\begin{document}
	\begin{CJK*}{UTF8}{gbsn}
		\maketitle
        \clearpage
        \section{拍卖}
        \subsection{拍卖的分类}
        \begin{itemize}
            \item 是否设定底价:分为有低价拍卖和无底价拍卖。
            \item 买卖双方是否同时参与竞价:分为单向拍卖和双向拍卖。
            \item 实际支付的买家数量:分为单一胜者支付拍卖和全支付拍卖。
            \item 竞价过程是否公开:分为公开拍卖(投标)和封标拍卖。
        \end{itemize}
        全支付拍卖:寻人启示、通缉悬赏。所有的投标人都要付出他们投标的费用或者代价,不论最后谁赢;单一价格拍卖,它是一种拍卖模式,
        一份拍卖商品的成交价只有一个,唯一的价格由参加投标的最高报价来确定。

        \subsection{拍卖的常见形式}
        \begin{itemize}
            \item 英国式:指在拍卖过程中,买者叫价,从低往高,直到无人加价为止。
            \item 荷兰式:指在拍卖过程中,卖者叫价,从高往低,直到有人要为止。
            \item 最高价封标拍卖:由买主在规定的时间内将密封的报价单递交给拍卖人,出价最高者或最低者中标。
            \item 第二高价封标拍卖:与最高价封标拍卖基本相同,区别仅在于中标者需要支付的价格是第二高的报价。
        \end{itemize}

        \subsection{维克里的研究}
        荷兰式拍卖和最高阶封标拍卖是策略性等价的,且在这两种拍卖方式下并不存在占优策略。英国式拍卖和第二高价封标拍卖也是策略性等价的,
        且在这两种拍卖方式下存在占优策略,即出价等于商品对竞拍者的私人价值。 即出价等于商品对竞拍者的私人价值,所谓私人价值就是指的一个人对物品的最高出价。
        无论卖家选择那种方式拍卖,其期望收益是相同的。这就是著名的收益等价定理。\par

        “收益等价定理”成立的条件:买卖双方风险中性,只有一件拍卖品;私人价值相互独立,买家之间不存在合谋;买家的评价有相同的分布,拍卖结构是共同知识。

        \clearpage
        \section{拍卖理论的若干成果}
        \subsection{若干成果 v1}
        给定买家数量,假定所有买家都是风险中性的,各个买家的私人价值相互独立,如果拍卖机制满足以下两个条件,拍卖的期望收益相等。
        \begin{enumerate}
            \item 最高价中标
            \item 出价最低的买家在不同拍卖机制下期望收益相同
        \end{enumerate}
        \subsection{若干成果 v2}
        \begin{itemize}
            \item 如果拍卖买家或卖家不是风险中性的,那么风险态度会影响拍卖结果。
            \item 英国式拍卖和第二高价封标拍卖,无论买家是风险中性还是风险厌恶,其拍卖结果不受影响。
            \item 荷兰式拍卖或最高价格封标拍卖,如果买家是风险厌恶的,会有一个更高的成交价。
            \item 如果卖家是风险厌恶的,选择最高价封标拍卖会更有利。
            \item 如果考虑买家合谋的情况下,第二高价封标拍卖和最高价封标拍卖相比更容易形成合谋。
        \end{itemize}
        \subsection{若干成果 v3}
        在同时存在共同价值和私人价值的情况下面,所谓共同价值,是指拍品有独立于私人价值的市场价值,比如100块,1克黄金。如果存在共同价值,
        买家的私人价值就会相互影响。在这种情况下,英国式拍卖会导致更高成交价。第二高价封标拍卖又比荷兰式和最高封标拍卖导致更高成交价。(关联原理)\par

        三级价格歧视:即对于同一商品,完全垄断厂商根据不同市场上的不同需求价格弹性,会制订不同的销售价格(钱大妈)。
        拍卖理论不仅仅局限于对竞价博弈的分析,还适用于那些不通过价格来进行资源分配的情况。

        \subsection{赢者诅咒}
        一件拍品只有私人价值,没有共同价值就不存在赢者诅咒。产生赢者诅咒,是既有私人价值又有共同价值的情况下面。比如一袋钱的最高估价猜测。
        当然赢者诅咒是双刃剑,对于卖家来说尽量给出足够多的拍品信息,以消除买家对赢者诅咒的担心。英国式相对不利于买家,荷兰式相对不利于卖家。

        \subsection{全支付拍卖求解}
        假设:卖品的共同价值为1,共N个参与竞拍。你的投标价为x且0<x<1
        \begin{itemize}
            \item[] 定义P(x)为参与人的投标价低于x的概率。
            \item[] 每个参与人出价x的概率为:$P(x) = x^{\frac{1}{N-1}}$
        \end{itemize}
        个人的最佳出价策略,随着参与人的越多,出价也就应该是越低。

        \clearpage
        \section{拍卖理论对股市投资的指导}
        从拍卖的视角,股市交易就是双向的拍卖博弈。买家相当于英国式拍卖的买家,要想中标必然要出全场的最高价,卖家相当于荷兰式拍卖的卖家,
        要想卖出股票必然要出全场最低的卖价。如果把当天的收盘价视作股票的共同价值,每笔成交价必然有一方是亏的。如果收盘价高于该成交价,卖家卖亏了。反之亦然。\par

        高开不断的上涨,根据贝叶斯法则,人们冲动性买入;如果低开,则修正自己的预期。股价的每次波动都是对股民的伤害,少看盘或不看盘。那么估值与盈利的比较:
        \begin{itemize}
            \item[] 高景气度策略:即买入当季业绩增速比较高的企业,这些往往都是当时最强势的品种。
            \item[] 低估值策略:买入PE或PB相比历史百分位更低的企业。
        \end{itemize}
        追求1年以内的相对收益,高景气度策略是最佳策略。追求3年以上的绝对收益,低估值策略是最佳策略。


    \end{CJK*}
\end{document}
